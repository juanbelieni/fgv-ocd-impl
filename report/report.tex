
% Setup
\documentclass[a4paper, 11pt]{article}


% Brazilian Portuguese
\usepackage[brazil]{babel}
\usepackage[utf8]{inputenc}
\usepackage[T1]{fontenc}

% Bibliography
% \usepackage[backend=biber,style=authoryear-icomp]{biblatex}

% Packages
\usepackage{graphicx} % Enables adding figures
\usepackage{subcaption} % Enables figures side-by-side
\usepackage{booktabs} % Makes nicer tables
\usepackage{geometry} % Let's you adjust the margins
\usepackage{setspace} % Let's you change the line spacing
\usepackage{hyperref} % Let's you add links (\url{<link>}) and clickable labels to naviguate through the document
\usepackage{microtype} % Improves the typesetting of your document by managing better the space between letters, hyphens etc.
\usepackage{parskip} % Improves the spacing between paragraphs
\usepackage{amsfonts} % Adds some math fonts

% Shortcuts
\newcommand{\R}{\mathbb{R}} % Shortcut for the set of real numbers
\newcommand{\eps}{\varepsilon} % Shortcut for the epsilon symbol

% Info
\author{Juan Belieni}
\title{Relatório: implementando o algoritmo de \textit{fuzzy c-means} em Python}
\date{\today}

% Document
\begin{document}

\maketitle
\onehalfspacing

% TODO: cite Bartak & Jastrzębska (2022)
\section{O algoritmo de \textit{fuzzy c-means}}

O algoritmo de \textit{fuzzy c-means} é um algoritmo baseado no tradicional algoritmo de \textit{k-means}.
Nele, cada ponto pode estar associado a mais de um cluster, com uma certa probabilidade, que deve somar 1.

No algoritmo, uma matriz de particionamento $U \in [0, 1]^{C \times P}$ é criada, onde $C$ é o número de clusters e $P$ é o número de pontos.
Cada elemento $u_{ij}$ da matriz representa a probabilidade do $i$-ésimo ponto pertencer ao $j$-ésimo cluster.
Também é definido um vetor $\mathbf v \in \mathbb R^{C \times M}$, que representa os centróides dos clusters, onde $M$ é a dimensão dos pontos.
A clusterização é controlada por um parâmetro $m \in [1, \infty)$, chamado de coeficiente de fuzzificação.

O \textit{fuzzy c-means} é um algoritmo iterativo, e busca minimizar a função objetivo

\begin{equation}
J = \sum_{i=1}^P \sum_{j=1}^C u_{ij}^m \| \mathbf z_i - \mathbf v_j \|^2,
\end{equation}

onde $\mathbf z_i$ é o $i$-ésimo ponto e $\mathbf v_j$ é o centro do $j$-ésimo cluster.
Para que essa minimização seja feita, o algoritmo define, a cada iteração, novos valores para a matriz de particionamento $U$ e para os centros dos clusters $\mathbf v$ da seguinte forma:

\begin{equation}
u_{ij} = \frac{1}{\sum_{k=1}^C \left( \frac{\| \mathbf z_i - \mathbf v_j \|}{\| \mathbf z_i - \mathbf v_k \|} \right)^{\frac{2}{m-1}}}.
\end{equation}

e

\begin{equation}
\mathbf v_j = \frac{\sum_{i=1}^P u_{ij}^m \mathbf z_i}{\sum_{i=1}^P u_{ij}^m}.
\end{equation}

A cada iteração, o algoritmo verifica se $J^{(k + 1)} - J^k < \eps$, onde $\eps$ é um parâmetro de tolerância.

\section{Implementação}

A implementação do algoritmo de \textit{fuzzy c-means} foi feita em Python, utilizando a biblioteca \textit{numpy} para a manipulação de matrizes e vetores.



\section{Métodos}

\section{Resultados}

\end{document}
